\documentclass[a4paper,onecolumn,oneside]{article}

\usepackage{hyperref}
\usepackage{parskip}
\usepackage{graphicx}
\usepackage{color}
\usepackage{subfigure}
\usepackage{amssymb}
\usepackage{amsmath}

\DeclareMathOperator{\avg}{avg}

\title{Distributed Systems\\Assignment 1}
\author{Clemens Wolff (s0942284)}
\date{March 13, 2014}

\begin{document}
\maketitle

\section*{Question 1}

Figure~\ref{fig:lattice} gives the lattice of consistent states $(p_1,p_2)$.

\begin{figure}[h]
\includegraphics[width=\textwidth]{lattice.pdf}
\caption{Lattice of consistent states.}
\label{fig:lattice}
\end{figure}

\section*{Question 2}

Starvation occurs when a thread that tried to enter its critical section (CS)
never actually did end up entering its critical section at any point in time.

An example of a system that leads to starvation is any system using the mutual
exclusion algorithm proposed by Dijkstra in \cite{dijkstra2001}.  The algorithm
works as follows.  When two processes attempt to enter the CS at the same time,
the process who most recently was in the CS is given priority.  This will lead
to starved processes if the same process is interested in the CS all the time
\cite[Theorem~5.1]{alagarsamy2003}.

Another system that leads to starvation is any system using the generalised
Peterson's mutual exclusion algorithm presented by \cite{hofri1990} with $N \geq
3$ processes.  The algorithm works as follows.  There are $N$ ``waiting rooms''
through which processes that wish to enter the CS have to advance.  Every
waiting room allows at least one process to advance and keeps one process
waiting. Assume there are three processes $p_1$, $p_2$ and $p_3$.  $p_2$
requests the CS, then $p_3$ requests the CS and then $p_1$ requests the CS\@.
Next assume that $p_1$ runs and then $p_3$ runs. $p_3$ now is eligible for the
CS again.  Assume that $p_3$ requests the CS again (and that $p_2$ requests the
CS after $p_3$).  $p_2$ and $p_3$ now take turns requesting the CS and running.
$p_1$ is starved \cite[Figure~1]{griffault2014}.

In order to take an example that is more related to this assignment, consider
the following system.  The Rickart-Agrawala mutual exclusion algorithm used in
the assignments gives process access to the CS by using a first-in-first-out
(FIFO) queue of critical-section request timestamps: process that request the CS
earlier get access first.  If the FIFO queue were to be replaced with a
last-in-first-out (LIFO) queue, the resulting system would lead to starvation.
In order to show this, consider the following example.  Two processes, $p_1$ and
$p_2$ want to access the CS\@.  $p_1$ sends its request before $p_2$ --- $p_2$
thus gets access to the CS\@.  Now assume that $p_2$ wants to access the CS
again, immediately after exiting it.  $p_2$ sends a CS request, has a lower
timestamp than $p_1$ and thus gets access.  Repeat.  $p_1$ is starved.

\section*{Question 3}

\textbf{Inputs:}  A distributed system represented by a graph $G=(V,E)$ where
$V$ are the nodes in the distributed system and $E$ are the connections between
the nodes.  A BFS tree $T$ of the graph rooted at some node $r$.

\textbf{Algorithm:}  The computation is started by some node sending a signal to
every other node in the system.  Without loss of generality, assume that the
root node $r$ initializes the computation.\footnote{If any non-root node wants
to initalise the computation, it can simply send a message to the root first.
The remainder of the algorithm works the same.  The time and message
complexities are not changed since, using the BFS tree, it takes at most $O(N)$
time and $O(N)$ messages to send the ``initialise computation'' message to the
root node.}  The root node $r$ sends an ``initialise computation'' message to
its children.  The children forward the message to their children recursively
until the leaves of the BFS tree are reached.  Every leaf node $p$ now sends its
value of $f.p$ to its parent $P^p$.  Every parent $P^j$ receives values
$f.p^1\ldots f.p^{k^{P^j}}$ where $k^{P^j}$ is the number of children of $P^j$.
The parents sends the values
$$\mathfrak{M}^{P^{P^j}} = \max\left\{f.p_1\ldots f.p_k^{P^j}\right\},$$
$$\mathfrak{S}^{P^{P^j}} = \sum\limits_{i=1}^{k^{P^j}} f.p_i\;\mathrm{and}$$
$$\mathfrak{K}^{P^{P^j}} = k + 1$$
to its own parent $P^{P^j}$.
Repeat until the root node $r$ of $T$ is reached.  The maximum value of $f$ in
the graph is the maximum of the values that the root node receives and the value
of $f$ at the root node:
$$\max\left\{p.f \mid p \in V\right\} = \max(r.f, \mathfrak{M}^r).$$
Similarly, the average value
of $f$ in the graph is:
$$\avg\left\{p.f \mid p \in V\right\} = {r.f + \mathfrak{S}^r \over \mathfrak{K}^r + 1}.$$
The root node now uses the BFS tree to distribute the maximum and average values
to the tree (i.e.\ the value is sent to the children of $r$ and those children
forward the value to their children recursively until the leaves of the tree are
reached).

\textbf{Time complexity:}  It takes at most $N$ time to send a message from
leaves to root in a BFS tree with $N$ items.  Initialising the computation and
collecting the tree-maximum tree-sum therefore take at most $O(N) + O(N) = O(N)$
time and distributing the values back in the tree takes at most $O(N)$ time.
Therefore the overall time complexity is $O(N)$.

\textbf{Message complexity:}  Every node in the tree only sends messages to its
children or parent.  This means that every node sends at most one message when
the initialisation message is sent or when collecting or distributing the
tree-maximum and tree-sum.  Therefore the overall message complexity is $3O(N) =
O(N)$.

\section*{Question 4}

The diameter of a graph is the longest shortest path in the graph (i.e.\ the
length of the shortest path between the most distant nodes).

Table~\ref{tbl:weighted-paths} shows that the diameter of the weighted graph is
8.  The path realising this diameter is $d\rightarrow h\rightarrow l\rightarrow
m$.

Table~\ref{tbl:unweighted-paths} shows that the diameter of the unweighted graph
is 5.  One path realising this diameter is $e\rightarrow a\rightarrow
c\rightarrow g\rightarrow l\rightarrow m$.  Another path realising this diameter
is $e\rightarrow a\rightarrow c\rightarrow g\rightarrow l\rightarrow m$.

\begin{table}[h]
\begin{subtable}
\centering
\begin{tabular}{c || c c c c c c c c c c c c}
Node & a& b& c& d& e& f& g& h& i& k& l& m\\
\hline \hline
a& 0& 2& 1& 4& 0& 2& 0& 3& 0& 3& 3& 5\\
b&  & 0& 2& 2& 2& 3& 1& 4& 1& 4& 4& 6\\
c&  &  & 0& 4& 1& 1&-1& 2&-1& 2& 2& 4\\
d&  &  &  & 0& 4& 4& 3& 3& 3& 6& 6& \colorbox{yellow}{8}\\
e&  &  &  &  & 0& 2& 0& 3& 0& 3& 3& 5\\
f&  &  &  &  &  & 0&-2& 1&-2& 1& 1& 3\\
g&  &  &  &  &  &  & 0& 3& 0& 3& 3& 5\\
h&  &  &  &  &  &  &  & 0& 3& 6& 3& 5\\
i&  &  &  &  &  &  &  &  & 0& 3& 3& 5\\
k&  &  &  &  &  &  &  &  &  & 0& 3& 2\\
l&  &  &  &  &  &  &  &  &  &  & 0& 2\\
m&  &  &  &  &  &  &  &  &  &  &  & 0\\
\end{tabular}
\caption{Length of shortest paths between nodes in weighted graph.}
\label{tbl:weighted-paths}
\end{subtable}
\qquad

\begin{subtable}
\centering
\begin{tabular}{c || c c c c c c c c c c c c}
Node & a& b& c& d& e& f& g& h& i& k& l& m\\
\hline \hline
a& 0& 1& 1& 2& 1& 2& 2& 3& 3& 4& 3& 4\\
b&  & 0& 1& 1& 2& 2& 2& 2& 3& 4& 3& 4\\
c&  &  & 0& 2& 2& 1& 1& 2& 2& 3& 2& 3\\
d&  &  &  & 0& 3& 3& 2& 1& 3& 3& 2& 3\\
e&  &  &  &  & 0& 3& 3& 4& 4& \colorbox{yellow}{5}& 4& \colorbox{yellow}{5}\\
f&  &  &  &  &  & 0& 2& 3& 1& 2& 3& 3\\
g&  &  &  &  &  &  & 0& 1& 1& 2& 1& 2\\
h&  &  &  &  &  &  &  & 0& 2& 2& 1& 2\\
i&  &  &  &  &  &  &  &  & 0& 1& 2& 2\\
k&  &  &  &  &  &  &  &  &  & 0& 1& 1\\
l&  &  &  &  &  &  &  &  &  &  & 0& 1\\
m&  &  &  &  &  &  &  &  &  &  &  & 0\\
\end{tabular}
\caption{Length of shortest paths between nodes in unweighted graph.}
\label{tbl:unweighted-paths}
\end{subtable}
\end{table}

\begin{thebibliography}{32}

\bibitem{dijkstra2001} Dijkstra, Edsger W. ``Solution of a problem in concurrent
programming control.'' \emph{Pioneers and Their Contributions to Software
Engineering.} Springer Berlin Heidelberg, 2001. 289-294.

\bibitem{alagarsamy2003} Alagarsamy, K. ``Some myths about famous mutual
exclusion algorithms.'' \emph{ACM SIGACT News} 34.3 (2003): 94-103.

\bibitem{hofri1990} Hofri, Micha. ``Proof of a mutual exclusion algorithm --- a
classic example.'' \emph{ACM SIGOPS Operating Systems Review} 24.1 (1990):
18-22.

\bibitem{griffault2014} Griffault, A., and G. Point. ``Proof of Peterson’s
algorithm for mutual exclusion.'' (2014).

\end{thebibliography}

\end{document}
